\documentclass[12pt]{article}
\newcommand\maintitle{
IEEE 802.15.4 TAP Link Type Specification
}
\newcommand\authorname{James Ko}
\newcommand\company[1][]{\textit{Exegin Technologies Limited}~}
\newcommand\versionnumber{1}
\title{
\maintitle
\\\company
\\\hfill\\\normalsize Version \versionnumber
}
\author{\authorname~\tt{<jck@exegin.com>}}

% We use literal underscores a lot -- disable it as the subscript indicator.
\catcode`\_=\active

% Comment out the line below when out of draft
\usepackage{draftwatermark}\SetWatermarkScale{3}
%\SetWatermarkText{Confidential}

\usepackage[none]{hyphenat}
\usepackage[margin=2cm]{geometry}
\usepackage{calc}
\usepackage{hyperref}
\usepackage{listings}
\usepackage{verbatim}
\usepackage{fancyvrb}
\usepackage{fancyhdr}
\usepackage{graphicx}
\usepackage{titlesec}
\usepackage{wrapfig}

%
% Bytefield and color customizations
%
\usepackage{bytefield}
\usepackage{color}

\definecolor{lightgray}{gray}{0.9}
\newcommand{\colorbitbox}[3]{%
    \rlap{\bitbox{#2}{\color{#1}\rule{\width}{\height}}}%
    \bitbox{#2}{#3}}

\pagestyle{fancy}
\fancyhead{}
\setlength{\headheight}{0pt}
\renewcommand{\headrulewidth}{0pt}
\renewcommand\_{\textunderscore\allowbreak}
\rfoot{}

\hypersetup{
  pdftitle=\maintitle,
  pdfauthor=\authorname,
  colorlinks=true,
  linkcolor=black,
  bookmarksnumbered=true
}

\usepackage{indentfirst}
\setlength{\parindent}{0pt}
\setlength{\parskip}{1.5ex}
\setlength{\footskip}{1cm}

\titleformat*{\subsubsection}{\Large\bfseries}

% Set the source-code listing style.
\lstset{
  basicstyle=\tt,
  showspaces=false,
  showstringspaces=false,
  numbers=left,
  frame=single,
  captionpos=b,
  breaklines=true,
  aboveskip=\baselineskip,
  belowskip=\baselineskip,
  framextopmargin=0pt,
  framexbottommargin=0pt,
  escapeinside={/@}{@/}
}

% Some useful formatting tags
% \texttt{terminal text}
% \textbf{bold}
% \texttt{\nameref{sec:<name>}}

\begin{document}
\begin{titlepage}
    \vspace{5cm}
    \begin{wrapfigure}{R}{0.25\textwidth}
        \includegraphics[scale=1]{exegin.png}
    \end{wrapfigure}
    \vspace{2cm}
    {\scshape\LARGE Exegin Technologies Limited \par}
    \vspace{5cm}
    {\scshape\Large IEEE 802.15.4 Information Link Type Specification\par}
    {\itshape Version \versionnumber}
    \vspace{2cm}
    \vfill

    \vfill

\begin{tabular}{ |p{3cm}|p{12cm}| }
\hline
\textbf{Revision}   &  \textbf{Description}\\
\hline
    2019-01-29 & Version 1 - Initial Release\\
\hline
\end{tabular}

    % Bottom of the page
    {\large \today\par}
\end{titlepage}

% Contents
\newpage
%\maketitle
\tableofcontents

\newpage
\section{Introduction}\label{sec:intro}
IEEE 802.15.4 is a wireless standard for Low-Rate Wireless Personal Area
Networks (LR-WPANs) defining a number of physical (PHY) layers covering a wide
variety of freqency bands and a number of Media Access Control (MAC) sub-layers
for managing data and management services including beacon management, channel
access, frame delivery and validation, and security mechanisms.  Developing,
debugging, diagnosing, and maintaining technologies using IEEE 802.15.4
require capturing packets with a sniffer. Sniffers generally output captured packets
encapsulated in a pcap or pcap-ng packet with a specific data link-type (DLT)
[\ref{itm:linktypes}] that is understood by packet analyzers.

Three existing DLTs for IEEE 802.15.4 are defined

\begin{itemize}
    \item DLT_IEEE802_15_4_WITHFCS (195),
    \item DLT_IEEE802_15_4_NONASK_PHY (215), and
    \item DLT_IEEE802_15_4_NOFCS (230).
\end{itemize}

None of the current DLTs provide a means to include out-of-band
meta-data such as received signal strength and channel number which are useful
for diagnostics in any wireless transmission system.  Also, the current
implementation with FCS only supports a 16-bit ITU-T CRC or CRC OK bit in the
status bytes provided by TI CC24xx radios, and the FCS type must be selected
by the user in the global protocol capture preferences.  The IEEE
802.15.4-2015 standard also specifies a 32-bit CRC equivalent to ANSI
X3.66-1979 for SUN PHYs or TVWS PHYs.  The global protocol configuration does
not support configurations with multiple sniffer interfaces each with a
different FCS type, 

This document proposes and defines a new DLT and format for presenting captured
IEEE 802.15.4 packets with meta-data to packet analyzers.

\newpage
\section{Overview}\label{sec:ovrvw}
The IEEE 802.15.4 TAP DLT design is based on an extensible Type-Length-Value
(TLV) format.  The optional meta-data information in encapsulated in one or
more TLVs and included in any packet from a sniffer.  Some types may be
duplicated by pcap-ng options but using the TLVs in this DLT enables providing
the meta-data information in both pcap and pcap-ng capture streams.  Where
types are duplicated by TLVs and pcap-ng options in the same packet,
the DLT TLVs have priority.

\subsection{PCAPNG Block Types}
Although DLTs are common to pcap and pcap-ng, this document will provide
examples using pcap-ng only.

\subsubsection{PCAPNG IDB}

The DLT value is specified in the pcap header or pcap-ng interface description
block (IDB) before the the first packet.

\begin{Verbatim}[samepage=true]
   0                   1                   2                   3
   0 1 2 3 4 5 6 7 8 9 0 1 2 3 4 5 6 7 8 9 0 1 2 3 4 5 6 7 8 9 0 1
  +-+-+-+-+-+-+-+-+-+-+-+-+-+-+-+-+-+-+-+-+-+-+-+-+-+-+-+-+-+-+-+-+
  |                Block Type = 0x00000001                        |
  +-+-+-+-+-+-+-+-+-+-+-+-+-+-+-+-+-+-+-+-+-+-+-+-+-+-+-+-+-+-+-+-+
  |                   Block Total Length                          |
  +-+-+-+-+-+-+-+-+-+-+-+-+-+-+-+-+-+-+-+-+-+-+-+-+-+-+-+-+-+-+-+-+
  |  DLT_IEEE802_15_4_TAP (283)   |        reserved               |
  +-+-+-+-+-+-+-+-+-+-+-+-+-+-+-+-+-+-+-+-+-+-+-+-+-+-+-+-+-+-+-+-+
  |                          SnapLen                              |
  +-+-+-+-+-+-+-+-+-+-+-+-+-+-+-+-+-+-+-+-+-+-+-+-+-+-+-+-+-+-+-+-+
  |                      Options (variable)                       |
  +-+-+-+-+-+-+-+-+-+-+-+-+-+-+-+-+-+-+-+-+-+-+-+-+-+-+-+-+-+-+-+-+
  |                      Block Total Length                       |
  +-+-+-+-+-+-+-+-+-+-+-+-+-+-+-+-+-+-+-+-+-+-+-+-+-+-+-+-+-+-+-+-+
\end{Verbatim}

\newpage
\subsubsection{PCAPNG EPB}

An IEEE 802.15.4 TAP packet is encapsulated in the packet data following a
pcap record header or in the packet data of a pcap-ng Enhanced Packet Block
(EPB).

\begin{Verbatim}[samepage=true]
   0                   1                   2                   3
   0 1 2 3 4 5 6 7 8 9 0 1 2 3 4 5 6 7 8 9 0 1 2 3 4 5 6 7 8 9 0 1
  +-+-+-+-+-+-+-+-+-+-+-+-+-+-+-+-+-+-+-+-+-+-+-+-+-+-+-+-+-+-+-+-+
  |                   Block Type = 0x00000006                     |
  +-+-+-+-+-+-+-+-+-+-+-+-+-+-+-+-+-+-+-+-+-+-+-+-+-+-+-+-+-+-+-+-+
  |                      Block Total Length                       |
  +-+-+-+-+-+-+-+-+-+-+-+-+-+-+-+-+-+-+-+-+-+-+-+-+-+-+-+-+-+-+-+-+
  |                         Interface_ID                          |
  +-+-+-+-+-+-+-+-+-+-+-+-+-+-+-+-+-+-+-+-+-+-+-+-+-+-+-+-+-+-+-+-+
  |                      Timestamp (High)                         |
  +-+-+-+-+-+-+-+-+-+-+-+-+-+-+-+-+-+-+-+-+-+-+-+-+-+-+-+-+-+-+-+-+
  |                       Timestamp (Low)                         |
  +-+-+-+-+-+-+-+-+-+-+-+-+-+-+-+-+-+-+-+-+-+-+-+-+-+-+-+-+-+-+-+-+
  |                    Captured Packet Length                     |
  +-+-+-+-+-+-+-+-+-+-+-+-+-+-+-+-+-+-+-+-+-+-+-+-+-+-+-+-+-+-+-+-+
  |                    Original Packet Length                     |
  +-+-+-+-+-+-+-+-+-+-+-+-+-+-+-+-+-+-+-+-+-+-+-+-+-+-+-+-+-+-+-+-+
  |                    IEEE 802.15.4 TAP Packet                   |
  |                variable length, padded to 32 bits             |
  +-+-+-+-+-+-+-+-+-+-+-+-+-+-+-+-+-+-+-+-+-+-+-+-+-+-+-+-+-+-+-+-+
  |                       Options (variable)                      |
  +-+-+-+-+-+-+-+-+-+-+-+-+-+-+-+-+-+-+-+-+-+-+-+-+-+-+-+-+-+-+-+-+
  |                      Block Total Length                       |
  +-+-+-+-+-+-+-+-+-+-+-+-+-+-+-+-+-+-+-+-+-+-+-+-+-+-+-+-+-+-+-+-+
\end{Verbatim}

\newpage
\subsection{IEEE 802.15.4 TAP Packet}

The IEEE 802.15.4 TAP Packet consists of the TAP Header, zero or more TLV
fields, the PHY payload (PSDU), and optional FCS bytes.  All data fields are
encoded in little-endian byte order.

\subsubsection{IEEE 802.15.4 TAP Header}

\begin{Verbatim}[samepage=true]
   0                   1                   2                   3
   0 1 2 3 4 5 6 7 8 9 0 1 2 3 4 5 6 7 8 9 0 1 2 3 4 5 6 7 8 9 0 1
  +-+-+-+-+-+-+-+-+-+-+-+-+-+-+-+-+-+-+-+-+-+-+-+-+-+-+-+-+-+-+-+-+
  |    version    |    padding    |            length             |
  +-+-+-+-+-+-+-+-+-+-+-+-+-+-+-+-+-+-+-+-+-+-+-+-+-+-+-+-+-+-+-+-+
  |                            TAP TLVs                           :
  :                        variable length                        |
  +-+-+-+-+-+-+-+-+-+-+-+-+-+-+-+-+-+-+-+-+-+-+-+-+-+-+-+-+-+-+-+-+
  |                              ...                              :
  :                   IEEE 802.15.4 PHY Payload                   :
  :                    [+ FCS per FCS Type TLV]                   :
  :                              ...                              |
  +-+-+-+-+-+-+-+-+-+-+-+-+-+-+-+-+-+-+-+-+-+-+-+-+-+-+-+-+-+-+-+-+
\end{Verbatim}

\begin{itemize}
    \item version - 0
    \item padding
    \item length - total length of header and TLVs in bytes
\end{itemize}

\subsubsection{IEEE 802.15.4 TAP TLVs}

IEEE 802.15.4 TAP TLVs have the following format.

\begin{Verbatim}[samepage=true]
   0                   1                   2                   3
   0 1 2 3 4 5 6 7 8 9 0 1 2 3 4 5 6 7 8 9 0 1 2 3 4 5 6 7 8 9 0 1
  +-+-+-+-+-+-+-+-+-+-+-+-+-+-+-+-+-+-+-+-+-+-+-+-+-+-+-+-+-+-+-+-+
  |             type              |             length            |
  +-+-+-+-+-+-+-+-+-+-+-+-+-+-+-+-+-+-+-+-+-+-+-+-+-+-+-+-+-+-+-+-+
  |                             value                             :
  :                variable length, padded to 32 bits             |
  +-+-+-+-+-+-+-+-+-+-+-+-+-+-+-+-+-+-+-+-+-+-+-+-+-+-+-+-+-+-+-+-+
\end{Verbatim}

\begin{itemize}
    \item type        - field type identifier
    \item length      - bytes in value field
    \item value       - data for type
\end{itemize}

\newpage
\section{IEEE 802.15.4 TAP TLV Types}

The following subsections describe the currently defined TLVs.  Any unknown
TLVs should be dissected as a binary blob of the provided length.

\subsection{FCS Type}

Identifies the FCS type following the PHY Payload.  FCS type none (0) is
optional if no FCS is present.

\begin{Verbatim}[samepage=true]
   0                   1                   2                   3
   0 1 2 3 4 5 6 7 8 9 0 1 2 3 4 5 6 7 8 9 0 1 2 3 4 5 6 7 8 9 0 1
  +-+-+-+-+-+-+-+-+-+-+-+-+-+-+-+-+-+-+-+-+-+-+-+-+-+-+-+-+-+-+-+-+
  |      Type = FCS_TYPE (0)      |           datalen = 1         |
  +-+-+-+-+-+-+-+-+-+-+-+-+-+-+-+-+-+-+-+-+-+-+-+-+-+-+-+-+-+-+-+-+
  |     Type      |             padding                           |
  +-+-+-+-+-+-+-+-+-+-+-+-+-+-+-+-+-+-+-+-+-+-+-+-+-+-+-+-+-+-+-+-+
\end{Verbatim}

The FCS type is one of
    \begin{itemize}
        \item 0 = None,
        \item 1 = 16-bit CRC,
        \item 2 = 32-bit CRC,
        \item 3 = TI CC24xx CRC_OK bit.
    \end{itemize}

\subsection{Receive Signal Strength}
The received signal strength in dBm as a IEEE-754 floating point number.

\begin{Verbatim}[samepage=true]
   0                   1                   2                   3
   0 1 2 3 4 5 6 7 8 9 0 1 2 3 4 5 6 7 8 9 0 1 2 3 4 5 6 7 8 9 0 1
  +-+-+-+-+-+-+-+-+-+-+-+-+-+-+-+-+-+-+-+-+-+-+-+-+-+-+-+-+-+-+-+-+
  |         Type = RSS (1)        |           datalen = 4         |
  +-+-+-+-+-+-+-+-+-+-+-+-+-+-+-+-+-+-+-+-+-+-+-+-+-+-+-+-+-+-+-+-+
  |                          RSS in dBm                           |
  +-+-+-+-+-+-+-+-+-+-+-+-+-+-+-+-+-+-+-+-+-+-+-+-+-+-+-+-+-+-+-+-+
\end{Verbatim}

\newpage
\subsection{Bit Rate}

The transmission data-rate in bps.  The bit-rate may change frame to frame in
multi-rate PHY configurations.

\begin{Verbatim}[samepage=true]
   0                   1                   2                   3
   0 1 2 3 4 5 6 7 8 9 0 1 2 3 4 5 6 7 8 9 0 1 2 3 4 5 6 7 8 9 0 1
  +-+-+-+-+-+-+-+-+-+-+-+-+-+-+-+-+-+-+-+-+-+-+-+-+-+-+-+-+-+-+-+-+
  |      Type = BIT_RATE (2)      |           datalen = 4         |
  +-+-+-+-+-+-+-+-+-+-+-+-+-+-+-+-+-+-+-+-+-+-+-+-+-+-+-+-+-+-+-+-+
  |                        Bit rate in bps                        |
  +-+-+-+-+-+-+-+-+-+-+-+-+-+-+-+-+-+-+-+-+-+-+-+-+-+-+-+-+-+-+-+-+
\end{Verbatim}

\subsection{Channel Assignment}

Channel assignments are defined through a combination of channel numbers and
channel pages.  See IEEE 802.15.4-2015 10.1.2 Channel assignments.
\begin{Verbatim}[samepage=true]
   0                   1                   2                   3
   0 1 2 3 4 5 6 7 8 9 0 1 2 3 4 5 6 7 8 9 0 1 2 3 4 5 6 7 8 9 0 1
  +-+-+-+-+-+-+-+-+-+-+-+-+-+-+-+-+-+-+-+-+-+-+-+-+-+-+-+-+-+-+-+-+
  | Type = CHANNEL_ASSIGNMENT (3) |           datalen = 3         |
  +-+-+-+-+-+-+-+-+-+-+-+-+-+-+-+-+-+-+-+-+-+-+-+-+-+-+-+-+-+-+-+-+
  |        Channel number         | Channel page  |    padding    |
  +-+-+-+-+-+-+-+-+-+-+-+-+-+-+-+-+-+-+-+-+-+-+-+-+-+-+-+-+-+-+-+-+
\end{Verbatim}

\subsection{SUN PHY Information}

An IEEE 802.15.4 SUN PHY is configured to operate in a specified frequency
band, encoding type, and rate mode.

\begin{Verbatim}[samepage=true]
   0                   1                   2                   3
   0 1 2 3 4 5 6 7 8 9 0 1 2 3 4 5 6 7 8 9 0 1 2 3 4 5 6 7 8 9 0 1
  +-+-+-+-+-+-+-+-+-+-+-+-+-+-+-+-+-+-+-+-+-+-+-+-+-+-+-+-+-+-+-+-+
  |   Type = PHY_ENCODING (4)     |         datalen = 3           |
  +-+-+-+-+-+-+-+-+-+-+-+-+-+-+-+-+-+-+-+-+-+-+-+-+-+-+-+-+-+-+-+-+
  |     Band      |     Type      |     Mode      |   padding     |
  +-+-+-+-+-+-+-+-+-+-+-+-+-+-+-+-+-+-+-+-+-+-+-+-+-+-+-+-+-+-+-+-+
\end{Verbatim}

\begin{itemize}
    \item Band - IEEE 802.15.4 Table 7-19 Frequency band identifer values.
    \item Type - IEEE 802.15.4 Table 7-20 Modulation scheme encoding values.
    \item Mode - IEEE 802.15.4 Rate mode depends on the Band and Type.
\end{itemize}

\newpage
\subsection{Start of Frame Timestamp}

The start of frame timestamp is a monotonically increasing time in nanoseconds
since power on of the receiving device.  This value differs from the timestamp
in the pcap or pcap-ng header which is based on the clock time reported by the
sniffer.

\begin{Verbatim}[samepage=true]
   0                   1                   2                   3
   0 1 2 3 4 5 6 7 8 9 0 1 2 3 4 5 6 7 8 9 0 1 2 3 4 5 6 7 8 9 0 1
  +-+-+-+-+-+-+-+-+-+-+-+-+-+-+-+-+-+-+-+-+-+-+-+-+-+-+-+-+-+-+-+-+
  |   Type = SOF_TIMESTAMP (5)    |          datalen = 8          |
  +-+-+-+-+-+-+-+-+-+-+-+-+-+-+-+-+-+-+-+-+-+-+-+-+-+-+-+-+-+-+-+-+
  |                                                               |
  |                          nanoseconds                          |
  |                                                               |
  +-+-+-+-+-+-+-+-+-+-+-+-+-+-+-+-+-+-+-+-+-+-+-+-+-+-+-+-+-+-+-+-+
\end{Verbatim}

\subsection{End of Frame Timestamp}

The end of frame timestamp is a monotonically increasing time in nanoseconds
since power on of the receiving device.  This value is important to
time-slotted MACs where a packet may overflow a time slot, or where there are
timing constraints on the ack sent in response to a data frame.

\begin{Verbatim}[samepage=true]
   0                   1                   2                   3
   0 1 2 3 4 5 6 7 8 9 0 1 2 3 4 5 6 7 8 9 0 1 2 3 4 5 6 7 8 9 0 1
  +-+-+-+-+-+-+-+-+-+-+-+-+-+-+-+-+-+-+-+-+-+-+-+-+-+-+-+-+-+-+-+-+
  |   Type = EOF_TIMESTAMP (6)    |          datalen = 8          |
  +-+-+-+-+-+-+-+-+-+-+-+-+-+-+-+-+-+-+-+-+-+-+-+-+-+-+-+-+-+-+-+-+
  |                                                               |
  |                          nanoseconds                          |
  |                                                               |
  +-+-+-+-+-+-+-+-+-+-+-+-+-+-+-+-+-+-+-+-+-+-+-+-+-+-+-+-+-+-+-+-+
\end{Verbatim}

\newpage
\subsection{Absolute Slot Number (ASN)}

For a Time-Slotted Channel Hopping MAC, the Absolute Slot Number (ASN)
is a monotonically increasing number which is synchronized across all nodes on
the network and forms part of the nonce for decryption.

\begin{Verbatim}[samepage=true]
   0                   1                   2                   3
   0 1 2 3 4 5 6 7 8 9 0 1 2 3 4 5 6 7 8 9 0 1 2 3 4 5 6 7 8 9 0 1
  +-+-+-+-+-+-+-+-+-+-+-+-+-+-+-+-+-+-+-+-+-+-+-+-+-+-+-+-+-+-+-+-+
  |        Type = ASN (7)         |          datalen = 8          |
  +-+-+-+-+-+-+-+-+-+-+-+-+-+-+-+-+-+-+-+-+-+-+-+-+-+-+-+-+-+-+-+-+
  |                                                               |
  |                          ASN (64-bits)                        |
  |                                                               |
  +-+-+-+-+-+-+-+-+-+-+-+-+-+-+-+-+-+-+-+-+-+-+-+-+-+-+-+-+-+-+-+-+
\end{Verbatim}

\subsection{Start of Slot Timestamp}

The start of slot timestamp is a monotonically increasing time in nanoseconds
since power on of the receiving device.  For a Time-Slotted Channel Hopping
MAC, the start of slot timestamp, which preceeds the start of frame timestamp,
is essential for debugging and optimizing TSCH configurations.

\begin{Verbatim}[samepage=true]
   0                   1                   2                   3
   0 1 2 3 4 5 6 7 8 9 0 1 2 3 4 5 6 7 8 9 0 1 2 3 4 5 6 7 8 9 0 1
  +-+-+-+-+-+-+-+-+-+-+-+-+-+-+-+-+-+-+-+-+-+-+-+-+-+-+-+-+-+-+-+-+
  |   Type = SLOT_TIMESTAMP (8)   |          datalen = 8          |
  +-+-+-+-+-+-+-+-+-+-+-+-+-+-+-+-+-+-+-+-+-+-+-+-+-+-+-+-+-+-+-+-+
  |                                                               |
  |                          nanoseconds                          |
  |                                                               |
  +-+-+-+-+-+-+-+-+-+-+-+-+-+-+-+-+-+-+-+-+-+-+-+-+-+-+-+-+-+-+-+-+
\end{Verbatim}

\subsection{Timeslot Length}

The timeslot length in a Time-Slotted Channel Hopping MAC is used for
calculating the delta between end of frame and end of slot.

\begin{Verbatim}[samepage=true]
    0                   1                   2                   3
    0 1 2 3 4 5 6 7 8 9 0 1 2 3 4 5 6 7 8 9 0 1 2 3 4 5 6 7 8 9 0 1
   +-+-+-+-+-+-+-+-+-+-+-+-+-+-+-+-+-+-+-+-+-+-+-+-+-+-+-+-+-+-+-+-+
   |   Type = TIMESLOT_LENGTH (9)  |          datalen = 8          |
   +-+-+-+-+-+-+-+-+-+-+-+-+-+-+-+-+-+-+-+-+-+-+-+-+-+-+-+-+-+-+-+-+
   |                 Length of timeslot in microseconds            |
   +-+-+-+-+-+-+-+-+-+-+-+-+-+-+-+-+-+-+-+-+-+-+-+-+-+-+-+-+-+-+-+-+
\end{Verbatim}


\newpage
\section{Glossary}\label{sec:glos}
\begin{center}
\begin{tabular}{ |p{3cm}|p{10cm}| }
\hline
\textbf{Notation}   &  \textbf{Description}\\
\hline
DLT                 &  Data-Link Type\\
\hline
EPB                 &  Enhance Packet Block (pcap-ng)\\
\hline
FCS                 &  Frame Check Sequence\\
\hline
IDB                 &  Interface Description Block (pcap-ng)\\
\hline
IEEE                &  Institute of Electrical and Electronics Engineers\\
\hline
LR-WPAN             &  Low-Rate Wireless Personal Area Network\\
\hline
MAC                 &  Medium Access Control\\
\hline
pcap                &  Packet Capture File Format [\ref{itm:libpcap}]\\
\hline
pcap-ng             &  PCAP Next Generation Capture File Format [\ref{itm:pcapng}]\\
\hline
PDU                 &  Protocol Data Unit\\
\hline
PHY                 &  Physical Layer\\
\hline
SUN                 &  Smart Utility Networks\\
\hline
TLV                 &  Type-Length-Value\\
\hline
TSCH                &  Time-Slotted Channel Hopping\\
\hline
TVWS                &  Television White Space\\
\hline
\end{tabular}
\end{center}

\section{References}\label{sec:references}
    \begin{enumerate}
        \item Data-Link Types - \url{https://www.tcpdump.org/linktypes.html}\label{itm:linktypes}
        \item Packet Capture File Format - \url{https://wiki.wireshark.org/Development/LibpcapFileFormat}\label{itm:libpcap}
        \item PCAP Next Generation Capture File Format - \url{https://pcapng.github.io/pcapng/}\label{itm:pcapng}
    \end{enumerate}
\end{document}
